\documentclass[a4paper, 12pt]{report}
 
\usepackage[utf8]{inputenc} % un package
\usepackage[T1]{fontenc}      % un second package
\usepackage[francais]{babel}  % un troisième package
\begin{document}

\chapter{StartAutomaton}
\section{Présentation du projet}
\subsection{Automates et expressions rationnelles}
\subsection{La bibliothèque Automaton}
\subsection{Sujet du projet}

\section{Fonctionnalités de StartAutomaton}
\subsection{Affichage}
\subsection{Tests}
\subsection{Sélecteurs et éditeurs}
\subsection{Manipulation d'automates}
\subsection{Conversions}

\section{Problèmes rencontrés}
\subsection{La bibliothèque Automaton}
\subsection{Méthodes de manipulation d'automates}
\subsubsection{Union et intersection}
\subsubsection{Déterminiser}
\subsubsection{Minimiser}
\subsubsection{Conversion d'une expression rationnelle vers un automate}
\subsubsection{Conversion d'une expression rationnelle en préfixée}

\section{Projection}
\subsection{Implémentation de la conversion d'une expression rationnelle en préfixée}
\subsection{Création de classes utilisables par StartAutomaton}

\section{Conclusion}
\subsection{Aboutissement du projet}
\subsection{Outils et méthodes de travail découvertes}

\end{document}